%%%%%%%%%%%%%%%%%%%%%%%%%%%%%%%%%%%%%%%%%%%%%%%%%%%%%%%%%%%%%%%%%%%%%%%%%%%%%%%%%%%%%%%
\PassOptionsToPackage{dvipsnames, svgnames}{xcolor} % paquetes para utilizar variedad de colores. ver más en http://en.wikibooks.org/wiki/LaTeX/Colors y http://www.latextemplates.com/svgnames-colors
\documentclass{beamer}  

\mode<presentation> {

% The Beamer class comes with a number of default slide themes
% which change the colors and layouts of slides. Below this is a list
% of all the themes, uncomment each in turn to see what they look like.

%\usetheme{default}
%\usetheme{AnnArbor}
%\usetheme{Antibes}
%\usetheme{Bergen}
%\usetheme{Berkeley}
%\usetheme{Berlin}
%\usetheme{Boadilla}
%\usetheme{CambridgeUS}
%\usetheme{Copenhagen}
%\usetheme{Darmstadt}
%\usetheme{Dresden}
%\usetheme{Frankfurt}
%\usetheme{Goettingen}
%\usetheme{Hannover}
%\usetheme{Ilmenau}
%\usetheme{JuanLesPins}
%\usetheme{Luebeck}
%\usetheme{Madrid}
%\usetheme{Malmoe}
%\usetheme{Marburg}
\usetheme{Montpellier}
%\usetheme{PaloAlto}
%\usetheme{Pittsburgh}
%\usetheme{Rochester}
%\usetheme{Singapore}
%\usetheme{Szeged}
%\usetheme{Warsaw}

% As well as themes, the Beamer class has a number of color themes
% for any slide theme. Uncomment each of these in turn to see how it
% changes the colors of your current slide theme.

%\usecolortheme{albatross}
%\usecolortheme{beaver}
%\usecolortheme{beetle}
%\usecolortheme{crane}
%\usecolortheme{dolphin}
%\usecolortheme{dove}
%\usecolortheme{fly}
%\usecolortheme{lily}
%\usecolortheme{orchid}
%\usecolortheme{rose}
%\usecolortheme{seagull}
%\usecolortheme{seahorse}
\usecolortheme{whale}
%\usecolortheme{wolverine}

%\setbeamertemplate{footline}               % To remove the footer line in all slides uncomment this line
%\setbeamertemplate{footline}[page number]  % To replace the footer line in all slides with a simple slide count uncomment this line

%\setbeamertemplate{navigation symbols}{}   % To remove the navigation symbols from the bottom of all slides uncomment this line
}
% % % % % % % % % % % % % % % % % % % % % % % % % % % % % % % % %
%user defined colors
\definecolor{grin}{RGB}{124, 179, 66}

% % % % % % % % % % % % % % % % % % % % % % % % % % % % % % % % % % %
\usepackage{booktabs}                       % Allows the use of \toprule, \midrule and \bottomrule in tables
\usepackage[utf8]{inputenc}                   % Para escribir tildes y eñes
\usepackage[spanish]{babel}
\usepackage{verbatim}	% Para texto plano
\usepackage{graphicx}     % Para insertar gráficas
\DeclareGraphicsExtensions{.pdf,.png,.jpg}
\usepackage{caption}
\usepackage{amsthm}
\usepackage{tikz}
\newtheorem{Teo}{Teorema}					% nuevo comando Teo, para escribir teoremas
%\usepackage{subcaption}

%%%%%%%%%%%%%%%%%%%%%%%%%%%%%%%%%%%%%%%%%%%%%%%%%%%%%%%%%%%%%%%%%%%%%%%%%%%%%%%%%%%%%%%
%---------------------------------  configuración de lisnting para código de C++
%
\usepackage{listings}	% Para códigos: C, C++, python, ver detalles en http://en.wikibooks.org/wiki/LaTeX/Source_Code_Listings
\lstset{language=C++, basicstyle=\color{DarkSlateGray}\ttfamily\footnotesize, keywordstyle=\color{RoyalBlue}\ttfamily\footnotesize, stringstyle=\color{Tomato}\ttfamily\footnotesize, commentstyle=\color{MediumSeaGreen}\ttfamily\footnotesize, morecomment=[l][\color{DarkViolet}]{\#}}

% ver variedad de colores en http://en.wikibooks.org/wiki/LaTeX/Colors y http://www.latextemplates.com/svgnames-colors
%------------------------------------------------


%----------------------------------------------------------------------------------------
%	TITLE PAGE
%----------------------------------------------------------------------------------------

\title[Modelado de Robots de 2 Ruedas]{{\tiny Proyecto Final Análisis de Sistemas}\\ IE-409} % The short title appears at the bottom of every slide, the full title is only on the title page

\author{Daniel García Vaglio\\Esteban Zamora Alvarado\\Ariel Fallas Pizarro} % Your name
\institute[UCR] % Your institution as it will appear on the bottom of every slide, may be shorthand to save space
{
Universdad de Costa Rica \\ % Your institution for the title page
\medskip
%\textit{correo@mail.com} % Your email address (optional)
}
\date{\today} % Date, can be changed to a custom date

\begin{document}

\begin{frame}
\titlepage % Print the title page as the first slide
\end{frame}

\begin{frame}
\frametitle{Resumen} 	% Table of contents slide, comment this block out to remove it
\tableofcontents 		% Throughout your presentation, if you choose to use \section{} and \subsection{} commands, these will automatically be %printed on this slide as an overview of your presentation
\end{frame}

%----------------------------------------------------------------------------------------
%	PRESENTATION SLIDES
%----------------------------------------------------------------------------------------

%------------------------------------------------
\section{Introducción} % Sections can be created in order to organize your presentation into discrete blocks, all sections and subsections are automatically printed in the table of contents as an overview of the talk
%--------------------------------------------------------

\begin{frame}
\frametitle{Robots de dos ruedas}
Robots diferenciales
\begin{figure}[h]
%\includegraphics[width=.7\textwidth]{../pictures/diagrama.eps}
\centering
\end{figure}
\end{frame}

%--------------------------------------------------------
\subsection{Marco Teórico}
\begin{frame}
\frametitle{Aplicaciones e Importancia}

\begin{columns}[c] % The "c" option specifies centered vertical alignment while the "t" option is used for top vertical alignment
	\column{.45\textwidth} % Left column and width
	\begin{figure}[h]
		%\includegraphics[width=\textwidth]{../pictures/importancia}
		\centering
        \end{figure}
	
	\column{.5\textwidth} % Right column and width
	\begin{figure}[h]
		%\includegraphics[width=.70\textwidth]{../pictures/aplicacion}
		\centering
	\end{figure}
\end{columns}
\end{frame}



%----------------------------------------------------------
\section{Modelo cinemático}
\subsection{Modelo Cinemático Directo}
\begin{frame}
	\frametitle{Modelo Cinemático Directo}
	\begin{columns}[c] % The "c" option specifies centered vertical alignment while the "t" option is used for top vertical alignment
		\column{.45\textwidth} % Left column and width
		
                \column{.5\textwidth} % Right column and width
	\end{columns}
\end{frame}

%----------------------------------------------------------
\subsection{Modelo cinemático inverso}

\begin{frame}
\frametitle{Modelo Cinemático Inverso}
		
\end{frame}

\subsection{Modelo Cinemático de punto desplazado}
\begin{frame}
\frametitle{Modelo Cinemático de Punto Desplazado}

\end{frame}

%-----------------------------------------------

\section{Modelo dinámico}
\begin{frame}
\frametitle{Modelo Dinámico}

\end{frame}

\begin{frame}
\frametitle{Modelo Dinámico}

\end{frame}
%------------------------------------------------

\section{Resultados}
\begin{frame}
\frametitle{Resultados de simulación}

\end{frame}
%························································
\begin{frame}
\frametitle{Resultados de simulación}

\end{frame}
%·······················································
\begin{frame}
\frametitle{Resultados de simulación}

\end{frame}


%-----------------------------------------------
\section{Conclusiones}
\begin{frame}
\frametitle{Conclusiones}

\end{frame}

%--------------------------------------------
\section{Preguntas}
\begin{frame}

\end{frame}

\end{document} 
