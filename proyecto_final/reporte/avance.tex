%Archivo de muestra de un Informe de pr�ctica o laboratorio
% utilizando la clase eieinforme.cls (V.M. Alfaro)
%---------------------------------------------------------------------

%pre�mbulo - uso de la clase eieinforme
\documentclass{eieinforme}
%
%--------- <Agregue aqu� alg�n paquete extra que sea necesario---------

%--------- Agregue aqu� alg�n paquete extra que sea necesario>---------
%
\begin{document}
\pagenumbering{roman}

%------------ Desde aqu� datos de la portada ------------------%

%t�tulo de la pr�ctica (* Por el usuario *)
\title{Avance del proyecto final}

%nombre del curso (* Por el usuario *)
\curso{IE-0409 An�lisis de Sistemas}

%nombre completo de los autores (* Por el usuario *)
\autorA{Ariel Fallas Pizarro, B4 ...}
\autorB{Daniel Garc�a Vaglio, B42781}
\autorC{Esteban Zamora Alvarado, B47769}

%fecha de presentaci�n (* si se deja en blanco indicar� la fecha de la compilaci�n)
%\date{Mes de a�o}

%---------------hasta aqu� datos de la portada-------------------------
%
%--------------------------------------------------------------------------------
%portada
\eietitlepage
\newpage
%--------------------------------------------------------------------------------
%
%---------------------------------------------------------------------------------
%------------- A partir de aqu� se escribe el informe ----------------------------
%---------------------------------------------------------------------------------
%las siguientes son las secciones iniciales del informe

%resumen

%-------------------
%tablas de contenido (* si alguna no se ocupa se puede ``comentar'' (%) para que no aparezca)
\newpage
\tableofcontents
\newpage
\listoffigures
\newpage
\listoftables
\newpage

%-------------------------------------------------------------------
%lista de la nomenclatura utilizada en el texto

\section*{Nomenclatura}

\begin{description}[labelindent=1cm,labelwidth=2.25cm,align=left]
%ponerlo en orden alfabetico
	\item[$J$] Inercia rotacional del robot.
	\item[$M$] Masa del robot.
	\item[$g$] Constante de la aceleraci�n de la gravedad.
	\item[$\theta(t)$] �ngulo del eje del robot respecto a la normal con el suelo.
	\item[$\omega(t)$] Velocidad angular.
        \item[$v$] Velocidad del centro de masa del robot.
        \item[$v_b$] Velocidad de la base del robot.
        \item[$\omega_r$] velocidad rotacional de las ruedas.
        \item[$$]
\end{description}

%--------------------
\newpage
\pagenumbering{arabic}
\pagestyle{fancy}
\setlength\headheight{14.5pt}
\lhead{\nouppercase{\leftmark}}
\rhead{\thepage}
\fancyfoot{}

%------------------------------------------------------------------
%Secciones del informe (puede ser un solo archivo o separarse los secciones en archivos independientes e incluirse con \include('nombre_archivo') si se quiere una nueva p�gina antes de la parte que se est� anexando o \input{'nombre de archivo'} si se quiere agregar directamente en la posici�n donde se llama a \input{}


%------------------
\section{Marco te�rico}

En esta secci�n se debe introducir toda la informaci�n que se crea conveniente para poner el trabajo en contexto. Por ejemplo, si se va a hablar de modelado de aerogeneradores, entonces se debe poner la historia, la construcci�n de los aerogeneradores, funcionamiento, etc.

\citep{alfaro2006b}


%-------------------
%bibliograf�a
\addcontentsline{toc}{section}{Bibliograf�a}
\bibliographystyle{EIEBib}
%nombre del archivo .bib que contiene la bibliograf�a
\bibliography{informe_ref}

%-------------------
%archivo de ap�ndice si fuera necesario (* si no hay ap�ndice comentar las siguientes dos l�neas *)

\addcontentsline{toc}{section}{Ap�ndices}
\section*{Ap�ndice A}

Datos o informaci�n complementaria.

%------------------
%final del documento
\end{document}
% Lo que se escriba a partir de aqu� ser� ignorado por el compilador